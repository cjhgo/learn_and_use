\documentclass[12pt]{article}
\usepackage{xeCJK}
\setCJKmainfont{KaiTi}

\setcounter{tocdepth}{4}
\setcounter{secnumdepth}{4}


\newenvironment{twenty}{%
	\begin{tabular*}{\textwidth}{@{\extracolsep{\fill}}ll}
}{%
	\end{tabular*}
}

\newcommand{\twentyitem}[4]{%
	\parbox[t]{0.83\textwidth}{%
		\textbf{#2}%
		\hfill%
		{\footnotesize#3}\\%
		#4\vspace{\parsep}%
	}\\
}

\begin{document}
\tableofcontents

\section{section}
\subsection{subsection}
\subsubsection{subsubsection}
\paragraph{paragraph}

\leavevmode\\

这是一行正常的文字这是一行正常的文字这是一行正常的文字\\
\begin{twenty}
  \twentyitem
  {yu}{d}{d}{d}
  \twentyitem
  {}{a}{a}{a}
\end{twenty}

{Lei Wang, Liangji Zhuang, \textit{\textbf{Junhang Chen}}, et al.\\}

$\cdots$


{
  line1\\
  \\
  line2
}

{\tiny 沉淀的}
\begin{huge}
  

\begin{flushleft}
  First (heading) line\\[\baselineskip]
  \leavevmode\\
  body of the centerd text...
\end{flushleft}
\end{huge}

{\fontsize{40}{48} \selectfont 发送到尽快发的是}
这里的的字号是多少

{\fontsize{12}{200} \selectfont 
发送到尽快发的是这里的的字号是多少\\
fdsjsdfkfds艰苦奋斗接口fdsjsdfkfds艰苦奋斗接口fdsjsdfkfds艰苦奋斗接口\\
fdsjsdfkfds艰苦奋斗接口fdsjsdfkfds艰苦奋斗接口
}


一行文字一行文字
shi kait 是开通吗
\bigskip
有一行\\

合理

不合理

      过故人庄\\

青山绕北郭
\end{document}