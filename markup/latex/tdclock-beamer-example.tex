\documentclass[10pt]{beamer}
%\usepackage[font=Times, timeinterval=10]{tickingclock}
%\usepackage[font=Cour, timeinterval=10]{tickingclock}
%\usepackage[font=Helv, timeinterval=30]{tickingclock}
\usepackage[font=Times,timeinterval=10, timeduration=2.0, timedeath=0, fillcolorwarningsecond=white!60!yellow,
timewarningfirst=50,timewarningsecond=80,resetatpages=2]{tdclock}
%\usepackage{times}
%\usepackage{helvet}

\usetheme{Madrid}


\title{An example of using Ticking Digital Clock in Beamer}
\author{Luis R\'{a}ndez \& Juan I. Montijano}
\institute[IUMA]{IUMA \\ Universidad de Zaragoza }
\date{\today}


\begin{document}

\frame{\titlepage}
%\pagereset{-1}

\begin{frame}[t]
\frametitle{Introduction}

\initclock  % This must be used one time to initialize the clock

We are putting here a clock with the macro  \quad
\structure{\texttt{$\backslash$tdclock}}:

\bigskip

\tdclock

\bigskip

Before that, the macro \quad  \structure{\texttt{$\backslash$initclock}}
\quad must have been used (only one time !).

\bigskip
This document uses the option \structure{\texttt{resetatpages=2}}. This means
that the crono is reset to zero at page 2.

Below you can see a crono advancing.  Every time you access to this page,
the crono should be reset to 0 (the output is updated every 10 seconds, as indicated
with the option \structure{\texttt{timeinterval=10}}, so it could appear
some non zero seconds, not greater than 10)

\bigskip
Crono: \quad\cronominutes\pdfcolon\cronoseconds

\bigskip

Wait some seconds until you see the crono advancing, then go to the next page, go back
to this page again and wait some seconds to see
how the crono is reset.

\end{frame}

% This is an example of inserting
% the clock at the place reserved to the date, the foot line in the
% case of Madrid theme.
%

\date{\tddate\ \ \tdtime}

\begin{frame}


Here we have inserted the clock at the place reserved to the date, the foot line in the case of Madrid theme.

\medskip

To do that, we have redefined the date contents with the order
\medskip

\centerline{\structure{\texttt{$\backslash$date\{$\backslash$tddate$\backslash$ $\backslash$ $\backslash$tdtime\}}} }

\medskip

Note that color and font size are adjusted according to the theme.

\medskip

Look at the foot!, the clock must be running.

\end{frame}

\begin{frame}


Next, we have inserted a button to toggle between clock and stopwatch,
and a button to reset the stopwatch to zero.

\medskip
\toggleclock{\beamerbutton{toggle}}
\tdtime
\resetcrono{\beamerbutton{reset}}

\medskip

Click at the left button, see what happens (you must wait 10 seconds),
then click at the right button and see the result.
Finally, click at the left button again to recover the current time

\bigskip

To do that, we have used

\medskip

\structure{
\texttt{$\backslash$toggleclock\{$\backslash$beamerbutton\{toggle\}\}} \\%
\texttt{$\backslash$tdtime} \\
\texttt{$\backslash$resetcrono\{$\backslash$beamerbutton\{reset\}\}}
}

\end{frame}

%
% Here we change the colors and size of the clock

\begin{frame}

Here we change the color and size of the clock with
standard \LaTeX commands.

\structure{\texttt{$\backslash$textcolor\{blue\}\{$\backslash$Large $\backslash$tdhours.$\backslash$tdminutes\}\}}}: %
\textcolor{green}{\Large \tdhours.\tdminutes}

\bigskip

Here a huge yellow clock showing only hours and minutes, inside a red box

\structure{$\backslash$hhmm $\backslash$factorclockfont\{6.0\} $\backslash$tdtime}

\begin{center}
\colorbox{red}{\color{yellow} \hhmm \factorclockfont{6.0} \tdtime}
\end{center}

\end{frame}

\begin{frame}

\centerline{Time-Date-Crono macros}
\begin{tabular}{|l|l|l|}
\hline
 \structure{$\backslash$initclock}        &                                      & initialize  clock   \\ \hline
 \structure{$\backslash$tickingclock}     & \tdclock                        & complete clock  \\ \hline
 \structure{$\backslash$tickingtime}      & \tdtime                         & current time    \\ \hline
 \structure{$\backslash$tickingdate}      & \tddate                         & current date    \\ \hline
 \structure{$\backslash$tickingday}       & \tdday                          & current day     \\\hline
 \structure{$\backslash$tickingmonth}     & \tdmonth                        & current month   \\\hline
 \structure{$\backslash$tickingyear}      & \tdyear                         & current year    \\\hline
 \structure{$\backslash$tickinghours}     & \tdhours                        & current hours   \\\hline
 \structure{$\backslash$tickingminutes}   & \tdminutes                      & current minutes \\\hline
 \structure{$\backslash$tickingseconds}   & \tdseconds                      & current seconds \\\hline
 \structure{$\backslash$crono}            & \crono                               & stopwatch       \\ \hline
 \structure{$\backslash$cronohours}       & \cronohours                          & crono hours   \\\hline
 \structure{$\backslash$cronominutes}     & \cronominutes                        & crono minutes \\\hline
 \structure{$\backslash$cronoseconds}     & \cronoseconds                        & crono seconds \\\hline
 \structure{$\backslash$resetcrono}       & \resetcrono{\beamerbutton{reset}}     & sets crono time to zero    \\\hline
 \structure{$\backslash$toggleclock}      & \toggleclock{\beamerbutton{toggle}}    & toggle time-crono   \\\hline
\end{tabular}

\end{frame}

\begin{frame}

\centerline{Formatting macros}
\begin{tabular}{|l|l|l|}
\hline
 \structure{$\backslash$hhmm}             & \hhmm\tdtime                    & format time   \\ \hline
 \structure{$\backslash$hhmmss}           & \hhmmss\tdtime                  & format time   \\ \hline
 \structure{$\backslash$mmddyyyy}         & \mmddyyyy\tddate                & format date   \\ \hline
 \structure{$\backslash$ddmmyyyy}         & \ddmmyyyy\tddate                & format date   \\ \hline
 \structure{$\backslash$dateseparator}    & \def\dateseparator{--}\tddate   & separator for date   \\ \hline
 \structure{$\backslash$timeseparator}    & \def\timeseparator{.}\tdtime    & separator for time   \\ \hline
 \structure{$\backslash$pdfslash}         & \pdfslash                            & slash in pdf font  \\ \hline
 \structure{$\backslash$pdfcolon}         & \pdfcolon                            & colon in pdf font   \\ \hline
 \structure{$\backslash$factorclockfont\{factor\}}     & \factorclockfont{2.0}\tdtime          & format date   \\ \hline
\end{tabular}

\end{frame}

\end{document}

